\label{Abstract}

\begin{center}

\Large
\textbf{Abstract}
    
\end{center}


La presente tesi si propone di esplorare e implementare soluzioni avanzate di gestione dei dati per migliorare l'efficienza e l'efficacia del reparto di assistenza tecnica di un'azienda. La ricerca è suddivisa in due fasi cruciali.

Nella prima fase, vengono analizzati approfonditamente i concetti chiave relativi ai Big Data, data warehousing, analisi dati e business intelligence. Questa fase di studio fornisce una base teorica solida per comprendere le sfide e le opportunità nel contesto aziendale.

La seconda fase si concentra sull'applicazione pratica dei concetti appresi durante lo studio iniziale. Dopo un dettagliato briefing con gli stakeholder aziendali, viene progettato e implementato un sistema di data warehousing e business intelligence personalizzato per il reparto di assistenza tecnica. L'obiettivo principale è quello di fornire agli operatori del reparto strumenti avanzati per analizzare dati in tempo reale, identificare tendenze e inefficienze, e prendere decisioni informate per ottimizzare i processi.

Il lavoro si propone di dimostrare come l'integrazione di tecnologie di data warehousing e business intelligence possa portare a miglioramenti significativi nelle operazioni quotidiane del reparto di assistenza tecnica, consentendo all'azienda di rispondere in modo più rapido ed efficace alle esigenze dei clienti. La tesi contribuisce quindi al progresso della gestione aziendale attraverso l'applicazione concreta di metodologie avanzate di analisi dati e business intelligence.