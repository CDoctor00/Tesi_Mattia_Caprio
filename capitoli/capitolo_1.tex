\chapter{Introduzione}
\label{ch:Introduzione}

\begin{citazione}
%La rivoluzione digitale è tale perché la tecnologia è divenuta un ambiente da abitare, una estensione della mente umana, un mondo che si intreccia con il mondo reale e che determina vere e proprie ristrutturazioni cognitive, emotive e sociali dell’esperienza, capace di rideterminare la costruzione dell’identità e delle relazioni. \~ Tonino Cantelmi \cite{tonino_cantelmi_cite}
Non esiste alternativa alla trasformazione digitale. Le aziende visionarie si ritaglieranno nuove opzioni strategiche, quelle che non si adatteranno falliranno. Jeff Bezos\cite{jeff_bezos_cite}
\end{citazione}

Ormai da anni il mondo ha inglobato nel proprio essere la tecnologia, rendendola parte di esso e in alcuni casi anche imprescindibile. L’evoluzione della tecnologia ha contribuito esponenzialmente alla produzione di nuovi sistemi che migliorassero la vita degli uomini, con automazioni e semplificazioni dei normai compiti tediosi o stancanti. Tale evoluzione ha comportato una sempre più importante evoluzione delle infrastrutture atte a permettere la creazione di tali sistemi e soprattutto di un ambiente interconnesso. Proprio per questo motivo, al giorno d'oggi molto spesso si da per scontato la tecnologia e la sua interazione con il mondo. Tuttavia, ciò non va sottovalutato, l'avvento della tecnologia fin dal primo istante ha comportato l'evoluzione del mondo intero dovuto alla necessità di adattamento a quest'ultima. Con il tempo ogni persona ha fatto parte di questo processo, utilizzando tale innovazione e migliorando il proprio stile di vita. Naturalmente, tale evoluzione ha condizionato anche il mondo aziendale, prendendo il nome di \textbf{trasformazione digitale}; più precisamente corrisponde al processo di evoluzione dei modelli aziendali, basato sugli avanzamenti tecnologici. Consiste in un cambiamento radicale attuato mediante strumenti digitali.~\cite{redhat_digital_transformation}

\section{Trasformazione Digitale}
Lo sviluppo economico deriva molto spesso da diversi cambiamenti sociali avvenuti nel tempo. La \textit{trasformazione digitale} (o \textit{digital tranformation, DT}) è uno degli esempi più importanti e recenti che manifestano tale cambiamento. Proprio per questo motivo, molti ricercatori hanno studiato approfonditamente questo fenomeno in modo da poterne identificare le possibili implicazioni, vantaggi, svantaggi e conseguenze sulle pratiche sociali e lavorative. Tutto allo scopo di sfruttare al massimo tale processo in modo da farlo diventare un'importante punto fondamentale che permetta di migliorare.~\cite{sciencedirect_digital_transformation}

In altre parole, è possibile affermare che la «la trasformazione digitale non è altro che il profondo cambiamento delle attività e dei processi organizzativi, delle competenze e dei modelli di business, che si effettua per sfruttare appieno, in modo strategico e prioritario, i cambiamenti e le opportunità che il mix di tecnologie digitale e il loro impatto accelerato hanno apportato alla società, avendo chiaro il percorso di cambiamento da implementare nell'organizzazione nel breve e nel lungo periodo.»~\cite{linkedin_digital_transformation_definition}

Secondo uno studio svolto nel 2018 è stato previsto che durante lo stesso anno sarebbero stati investiti circa 1,3 miliardi di dollari da parte delle aziende per applicare tecniche digitali in modo da migliorare l'efficienza, aumentare il valore del cliente e creare nuove opportunità di monetizzazione. Tuttavia, tale trasformazione non è affatto un compito semplice; infatti, sempre secondo la ricerca, il 70\% di questi processi non riuscirà nel proprio intento, comportando  una perdita totale di oltre 900 milioni di dollari.~\cite{forbes_digital_transformation_fail}

\subsection{Strategie della trasformazione digitale}

A seguito di approfonditi studi e dell'incremento dell'interesse in questo mondo, molte aziende negli ultimi anni hanno intrapreso un processo di esplorazione e innovazione adoperando nuove tecnologie digitali così da poterne sfruttare i benefici. Ciò ha comportato la necessità di svolgere procedimenti atti a trasformare anche le principali operazioni svolte dalle aziende in questione. Però svolgere tale procedura è necessario formulare una strategia di trasformazione corretta e coordinata.~\cite{digital_transformation_strategies}

\subsection{Benefici}

La maggior parte dei benefici della trasformazione digitale nell'ambito possono essere suddivisi in cinque differenti gruppi:~\cite{sciencedirect_digital_transformation_benefits}
\begin{itemize}
    \item Miglioramento della produttività: i processi di sviluppo e progettazione hanno diminuito le relative tempistiche necessarie per essere eseguiti.
    \item Incremento della qualità: tramite apposite misurazioni ad alta risoluzione dei parametri di produzione e dei prodotti durante l'interno processo di produzione è possibile comprendere al meglio tutte le sue fasi da conoscere su quale di questi svolgere delle azioni che comportino un risparmio.
    \item Maggiore personalizzazione dei prodotti: svolgendo analisi sui dati rispetto alle interazioni con gli utenti e ulteriori analisi di mercato è possibile comprendere quale siano le preferenze del proprio pubblico.
    \item Aumento della sicurezza: naturalmente l’evoluzione digitale porta non solo migliorie in ambito economico, ma anche in quello della sicurezza del singolo utente che, grazie a nuovi strumenti e metodologie, può svolgere in maggiore sicurezza (o magari non farlo proprio) compiti pericolosi, venendo preventivamente avvisato di potenziali rischi e su possibili contromisure.
\end{itemize}

\section{Analisi dei dati}

L'avvento del processo di trasformazione digitale ci permette di dire che ormai qualsiasi tecnologia è associabile ai dati che essi creano o comportano con la loro esistenza, ed essendo il nostro un mondo intriso e fondato sulla tecnologia i dati e le informazioni non possono che essere una delle risorse più importanti.

\begin{quote}
    «Le tecnologie aprono orizzonti incredibili. Internet non dà informazione, dà soltanto dati. L'informazione si ha quando i dati sono stati assimilati dal cervello.»
\end{quote}

Queste sono le parole di Federico Faggin\cite{federico_faggin_cite}, fisico e imprenditore italiano che ha preso parte in prima persona allo sviluppo dell’era dei computer come sono intesi oggi. Le sue parole sono perfette per riassumere il mondo al giorno d’oggi, ovvero un sistema basato sulla conoscenza derivata delle informazioni ricavate da Internet e dal Web. Proprio per questo motivo, le più grandi aziende ormai fondano il proprio business sull’analisi dei dati. La conoscenza è potere e come tale ha un valore inestimabile, ma per quanto ormai esistano una miriade di dati, questi non potranno mai diventare tale se non diventino in primis un’informazione analizzabile. È su questa idea che si basa il principio dell'analisi dei dati.

L’analisi dei dati permette che questi diventino un’informazione fruibile ed utile, la Business Intelligence (BI) e la Business Analytics (BA) permettono di evolvere questo principio per permettere di recuperare i dati, gestirli e mostrarli nel miglior modo possibile tale da permettere una analisi accurata dei dati in questione. Al fine di poter svolgere il proprio compito al meglio, le aziende di ogni settore cercano di sfruttare al meglio ogni dato a loro disposizione per migliorare ed incrementare i propri prodotti e servizi.

\section{Obiettivo della tesi}

\section{Struttura della tesi}