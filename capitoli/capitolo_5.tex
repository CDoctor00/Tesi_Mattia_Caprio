\chapter{Conclusioni e Sviluppi Futuri}
Il progetto presentato si è concentrato sul creare un sistema di data warehousing per la raccolta, gestione ed infine analisi dei dati generati dal reparto interno adibito al supporto tecnico dei clienti. L'obiettivo dello stesso ha riguardato il miglioramento delle logiche applicate al processo aziendale relative all'assistenza fornita dal reparto in questione. Per completare tale obiettivo, il progetto è stato quindi suddiviso in due fasi differenti: uno studio preliminare e una progettazione del sistema.

\subsection{Studio preliminare}

Nella prima fase del progetto, è stato condotto uno studio approfondito sui Big Data, il data warehousing, sull'analisi dei dati e la business intelligence. Per tale motivo sono stati approfonditi i vari metodi e tecniche utilizzate in tali ambiti, con annessa analisi sulle architetture dei sistemi in questione.

Attraverso tale studio è stato possibile acquisire una comprensione vasta e solida dei principi fondamentali degli argomenti trattati. Ciò ha permesso di identificare nel data warehousing e nella business intelligence le tecniche di gestione ed analisi dati più consone allo scopo del progetto stesso. Sono state così definite le strategie che avrebbero potuto migliorare le logiche aziendali comportando maggiore efficienza e qualità nel processo aziendale. 

\subsection{Progettazione del sistema}

Nella seconda fase del progetto, sono state applicate le conoscenze acquisite durante lo studio preliminare per progettare un sistema di data warehousing che potesse occuparsi della raccolta e gestione di tutti i dati generati dal reparto di assistenza tecnica. Questo sistema è stato progettato allo scopo di centralizzare e organizzare tutti i dati sui cui sviluppare un sistema di business intelligence. Ciò ha permesso di comprendere le criticità dell'attuale processo di gestione del reparto e poter così analizzare eventuali soluzioni che potessero migliorare efficienza ed efficacia.

Il sistema così progettato, sfruttando un apposito processo di ETL, ha permesso di svolgere avanzate e specifiche tecniche di analisi dei dati grazie ad un'interfaccia di business intelligence per fornire insight importanti per lo sviluppo del reparto di assistenza. Inoltre, per come è stato definito, il sistema potrebbe essere adottato dall'azienda anche con lo scopo di aiutare a comprendere in modo più tempestivo eventuali problemi e risolverli di conseguenza, migliorando così la soddisfazione della propria clientela.

\subsection{Conclusioni finali}

In conclusione, grazie ad un sistema progettato ad hoc sulle esigenze precedentemente definite, l'analisi ha coinvolto la raccolta, la pulizia e la trasformazione dei dati esistenti, consentendo di identificare le tendenze, modelli e criticità nel funzionamento del reparto di assistenza tecnica. L'utilizzo di strumenti di business intelligence, che mettessero a disposizione statistiche e visualizzazioni delle informazioni così ricavate, ha facilitato la comprensione delle dinamiche sottostanti e ha fornito una base solida per la gestione del relativo processo aziendale. La combinazione di queste tecnologie fornisce una base solida per la presa di decisioni informate, consentendo all'azienda di rimanere competitiva e orientata al futuro.

In altre parole, questo progetto ha dimostrato l'importanza che un sistema di gestione ed analisi dati può avere nel miglioramento delle logiche aziendali. Attraverso uno studio approfondito e una progettazione minuziosa di un sistema basato sulle necessità dell'azienda, è possibile proporre soluzioni concrete che permettono di aumentare l'efficienza, l'efficacia, la qualità e la sicurezza delle prese di decisioni da parte dei manager aziendali. L'implementazione del sistema ha portato a miglioramenti nelle logiche aziendali. La capacità di monitorare in tempo reale le richieste dei clienti, anticipare le esigenze e ottimizzare le risorse ha contribuito ad una maggiore efficienza operativa del reparto in questione. La BI ha fornito strumenti per l'identificazione proattiva dei problemi e la pianificazione strategica delle risorse. Questo perché la messa a disposizione delle informazioni chiare e dettagliate ha permesso ai manager aziendali di prendere decisioni informate e guidate dai dati strutturati.

Tuttavia, è importante sottolineare che la progettazione e l'implementazione di un sistema così definito è un processo complesso che richiede una pianificazione accurata e una corrispondente efficace gestione. Se tale requisiti non venissero soddisfati, il risultato del progetto stesso potrebbe riportare risultati negativi da ogni punto di vista, comportando a questo punto il rischio di compromettere il processo decisionale che si vorrebbe migliorare. Inoltre, il successo di un tale sistema, come spesso accennato durante lo studio, dipende in gran parte dalla qualità dei dati recuperati e messi poi a disposizione e dalle capacità dell'azienda di sfruttare efficacemente le informazioni derivate dall'analisi degli stessi.

\subsection{Considerazioni su possibili sviluppi futuri}

Sarebbe di particolare interesse considerare la possibilità di esplorare nuovamente tale progetto in un prossimo futuro, con l'ambizioso obiettivo di perfezionare ulteriormente le tecniche già implementate o, alternativamente, di adottare nuove tecnologie che si siano state sviluppate nel corso del tempo. Ciò consentirebbe di apportare miglioramenti significativi sia al processo di ETL che all'analisi dei dati, promuovendo un costante raffinamento delle metodologie operative. Inoltre, sarebbe intrigante valutare la possibilità di evolvere ed espandere il progetto, estendendo la sua portata a tutti i processi decisionali aziendali. Questo ampliamento strategico potrebbe contribuire in maniera sostanziale allo sviluppo complessivo dell'azienda, offrendo un impatto positivo su diversi settori e contribuendo a elevare l'efficienza operativa a livello globale. Tale approccio riflette l'aspirazione a partecipare attivamente all'evoluzione tecnologica e all'ottimizzazione continua dei processi decisionali, con l'obiettivo ultimo di consolidare il posizionamento dell'azienda come protagonista nell'ambito delle migliori pratiche industriali e delle soluzioni innovative.